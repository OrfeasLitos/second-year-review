\section{Plan of action}
  \subsection{Goals for following year}
    My primary goal at present is to complete and submit the Payment Network
    paper. A number of other projects are under consideration. These include the
    following:
    \begin{itemize}
      \item A combination of content curation with off-chain transactions. The
      aim of this project is to create a social networking blockchain-backed
      platform similar to Steemit, but where not every vote interacts with the
      blockchain. This will enable scalability for the particular application of
      decentralised content curation.
      \item Completion of the project of simplifying the Ledger functionality.
      The aim would be to create a functionality that is equivalent to the
      current form of the Ledger functionality (as found in 'Ouroboros
      Genesis'~\cite{genesis}), but which would achieve a cleaner separation of
      the application and the consensus layer of ledgers. The usecase for this
      functionality would be for researchers that are interested in creating new
      blockchain applications but prefer not to understand in depth how the
      consensus layer works, but use a simple abstraction for it instead.
      \item Classification of decentralised content curation mechanisms. As we
      learned while working on Steemit, designing such mechanisms is no easy
      task. The solutions used in conventional centralised social networks are
      not directly applicable to the decentralised setting. An interesting
      frontier to pursue would be a characterisation of the design space of such
      algorithms. An interesting result of such research would be to reach some
      form of impossibility result with respect to what quality of curation is
      possible. Taking this a step further, we could collaborate with machine
      learning experts to look for satisfactory decentralised content curation
      mechanisms.
      \item Creation of virtual payment channels on Bitcoin~\cite{bitcoin}. A
      virtual payment channel is a channel that can be opened and closed without
      touching the blockchain; It uses two pre-existing payment channels as its
      'blockchain'. This enables off-chain transactions between Alice and
      Charlie, even though they do not have an on-chain channel. If they both
      have an on-chain channel with Bob, they first ask for the help of Bob to
      create a virtual Alice - Charlie channel that 'sits on top of' the Alice -
      Bob and the Bob - Charlie channels. Subsequently Alice and Charlie can
      perform off-chain payments to each other without requiring the
      intermediation of Bob for every single transaction. This construction is
      particularly hard (i.e. interesting) to achieve in Bitcoin, given the very
      limited, non-Turing-complete scripting language it uses.
      \item Design of a new blockchain that natively enables payment channels.
      Most existing fully decentralised blockchains were designed before the
      severe issue of scalability was discovered. Their design implicitly
      assumes that all transactions are to be put on-chain. As a result, most
      existing 2nd layer solutions use compatibility tricks and complex
      techniques that are hard to analyse and verify, not to mention the
      overhead they incur. There is a lack of research yet on how to design a
      blockchain from first principles that embraces the concept of 2nd layer
      solutions. This is a topic that inspires me, given that it can help
      blockchains achieve their full potential.
    \end{itemize}

    Of course it is impossible to achieve all the aforementioned goals in the
    next year. A more realistic goal would be to complete or have satisfactory
    progress in 3 of the projects described. I will consider a project as
    completed once a relevant paper that contains a satisfactory degree of
    results has been accepted in a respected conference.

  \subsection{Thesis completion plan}
    I aim to submit the Lightning Payment Network paper until the end of May
    2019. Subsequently I would like to devote at most 6 months to writing and
    submitting a paper on designing virtual payment channels on Bitcoin. In the
    meanwhile I would like to pursue the Ledger functionality project with the
    aim of publishing another paper. After that, I plan to dedicate the rest of
    my studentship to the following two projects: that of classifying content
    curation systems and that of designing a payment network-enabled blockchain.

    I expect that my thesis will be on formalising, securing and optimising 2nd
    layer payment networks. Given on the one hand that it is a new area of
    blockchain research and is likely to be the main means through which
    blockchains will manage to scale to the needs of a global market, and on the
    other hand that this past year I have thoroughly understood several
    approaches to their construction and analysis, I believe that choosing this
    as the topic of my thesis is both a realistic target and a possibly
    influential and useful result for both the blockchain community and the
    general population.
