\section{Progress to date}
  \subsection{Work during past year}
    This year I have steadily increased my knowledge in multiple fronts, as well
    as started dedicating more and more fruitful time to producing novel
    research. To begin with, I have studied content curation, a concept that
    refers to methods and techniques used to evaluate and rank user-generated
    content in social media platforms and online forums. Our work on this
    subject culminated in the authoring and successful publication of 'A Puff of
    Steem: Security Analysis of Decentralized Content
    Curation'~\cite{puffofsteem}. In this paper we define an abstract system
    within which one can express a wide variety of concrete curation mechanisms.
    Furthermore, we leverage this system to analyse the quality of content
    curation of the Steemit\footnote{\url{https://steemit.com}} platform. This
    work exemplifies the lack of quality decentralised content curation
    mechanisms by highlighting the problems that arise by the use of the
    particular curation method employed by Steemit.

    Furthermore, I have obtained a solid (but not yet complete) understanding of
    the 'Universal Composability'~\cite{canetti2001universally} framework, which
    is a cryptographic model designed to facilitate proving secure a
    cryptographic protocol that runs concurrenly with other protocols. This
    makes proofs and protocols composable in the sense that a protocol that has
    been proven UC-secure can be run securely on a system that runs many
    independent copies of this protocol and other arbitrary algorithms and
    protocols, not known at the time of proof.

    In parallel I have studied several 2nd layer blockchain solutions, a set of
    techniques that aim to increase the scalability of blockchains by allowing
    multiple transactions to happen without having to add them to the
    blockchain. In particular, two parties that want to financially interact
    with each other often but do not trust each other can lock some of their
    funds in a new channel by adding a single transaction to the blockchain,
    transact an unlimited number of times within this channel off-chain and
    finally unlock the funds with the balance that corresponds to the latest
    state of the channel with another on-chain transaction. The knowledge gained
    through this study is currently being used in order to model and prove
    secure the 'Lightning Payment Channels'~\cite{lightning}, the most widely
    used 2nd layer solution, within the UC framework. I am currently close to
    the completion of the proof of security and subsequently of this project.

    Another related topic of interest for me has been understanding the
    blockchain itself. In this subject, I have studied several formalisations of
    blockchains (e.g. 'Bitcoin Backbone'~\cite{garay2015bitcoin}) and particular
    blockchain constructions (e.g. 'Ouroboros'~\cite{kiayias2017ouroboros}) and
    implementations (e.g. 'bcoin.js'\footnote{\url{https://bcoin.io}}). My
    contribution consists of an ongoing project to design a new functionality
    that describes the requirements that any distributed ledger should fulfill
    in the simplest form possible.

    Moreover, I have been aquainted with several cryptographic primitives and
    have developed and proven secure new ones to aid the formalisation of
    Payment Channels. Through personal and group work, I have amassed knowledge
    on low-level symmetric encryption primitives and hash functions and I
    understand the majority of concepts connected to the paradigm of
    simluation-based security.

    Last but not least, I have developed my teaching and transferrable skills
    through creating exercises for other students and delivering presentations
    in conferences, workshops and meetings.

  \subsection{Meetings with supervisor}
    Since 1/5/2018, we have had approximately one meeting per 10 days. The vast
    majority lasted for one hour and were one-to-one. These meetings have
    greatly helped me resolve questions with regards to the cryptographic
    methodology in general, simulation-based security, and the UC framework
    usage in particular. They helped me gain intuition on why certain approaches
    are taken and enabled me to develop my reasoning techniques with respect to
    them. Furthermore, Prof. Kiayias provided me with invaluable modelling ideas
    that catalysed the process of achieving our research objectives.
    Additionally, the Cryptography and Security group (headed by Prof. Kiayias)
    meets every week. Members of the group report on their progress there and
    interesting, informative discussions take place.

  \subsection{Conferences, seminars, workshops}
    I have attended or am attending the following:
    \begin{itemize}
      \item Real World Crypto
      2019\footnote{\url{https://rwc.iacr.org/2019/index.html}}
      \item Summer School on real-world crypto and privacy
      2019\footnote{\url{https://summerschool-croatia.cs.ru.nl/2019/}}
      \item Security and Privacy group weekly seminar
      \item Symmetric Cryptography weekly workshop
      \item IOHK Miami Summit 2019\footnote{\url{https://iohksummit.io/}}
    \end{itemize}

  \subsection{Achievements}
    This is a list of my achievements for the past year:
    \begin{itemize}
      \item Gave a presentation on the basics of Bitcoin for SIGCoin in the
      University of Edinburgh.
      \item Gave an invited talk at 'Bitcoin Wednesday
      Amsterdam'\footnote{\url{https://www.bitcoinwednesday.com/events/bitcoin-wednesday-65/}}
      on 'Trust is Risk'.
      \item Gave a presentation at IOHK Miami Summit
      2019\footnote{\url{https://iohksummit.io/}} on 'Lightning Payment
      Channels'.
      \item Was accepted and gave a presentation at Tokenomics
      2019\footnote{\url{http://tokenomics2019.org/}} on 'A Puff of Steem:
      Security Analysis of Decentralized Content Curation'.
      \item Improved my teaching skills in my role as Tutor for the Computer
      Security course.
      \item Improved my skills in creating assignments for students in my
      Teaching Assistant role for the Computer Security course.
      \item Learned more LaTeX, JavaScript, Python and gained general
      programming knowledge.
      \item Learned the internals of Bitcoin.
      \item Contributed to the bcoin.js library.
      \item Contributed to \url{https://blockchain-course.org/}.
      \item Contributed to Prof. Kiayias lecture notes on Cryptography.
      \item Improved my knowledge on computer networks.
    \end{itemize}
